\documentclass{resume}

\usepackage[left=0.5in,top=0.5in,right=0.5in,bottom=0.6in]{geometry} % Document margins
\newcommand{\tab}[1]{\hspace{.2667\textwidth}\rlap{#1}}
\newcommand{\itab}[1]{\hspace{0em}\rlap{#1}}
\usepackage[most]{tcolorbox}

\makeatletter
\def \printname {
  \begingroup
    \hfil{\namesize\bfseries \@name}\hfil
    \nameskip\break
  \endgroup
}
\makeatother

% Your address

\address{ \href{http://www.linkedin.com/in/juan-diego-heredia-nino}{\underline{{LinkedIn}}} $|$ +57 (316) 043-5259 $|$  \href{http://jdheredian.github.io}{\underline{Página web}} $|$ \href{mailto:juandiegoheredianino@gmail.com}{\underline{juandiegoheredianino@gmail.com}}}
\name{Juan Diego Heredia Niño} % Your name


\begin{document}

%----------------------------------------------------------------------------------------
%	EDUCATION SECTION
%----------------------------------------------------------------------------------------
\begin{rSection}{Perfil}
Estudiante de la Maestría en Economía y economista de la Universidad de los Andes. Intereses orientados a la aplicación de métodos cuantitativos, como estadística y aprendizaje de maquinas, en investigación, tanto académica como operacional. Temas de interés centrados en análisis/ciencia de datos, macro/microeconomía aplicada y aprendizaje de máquinas. Capacidad para trabajar en equipo de manera organizada, resolver problemas bajo presión, manejar grandes volúmenes de datos con precisión y atención al detalle. Orientación al servicio implementando habilidades comunicativas efectivas.
\end{rSection}
%----------------------------------------------------------------------------------------
%	WORK EXPERIENCE SECTION
%----------------------------------------------------------------------------------------
\begin{rSection}{Experiencia Laboral}
{\bf Asistente de investigación – Fedesarrollo} \hfill {\em Marzo, 2025 – Actualidad}
\begin{itemize}
    \item Realizo investigación empírica sobre oportunidades y desafíos del nearshoring para Colombia en sectores estratégicos de exportación.
    \item Construí bases de datos de comercio internacional (exportaciones e importaciones desde 1996) con UN COMTRADE e ITC (International Trade Centre) utilizando Python.
    \item Creé robots automatizados con técnicas de web scraping para extraer proyecciones de exportaciones desde la herramienta online del ITC, lo cual mejoró la eficiencia del cálculo y sistematización de resultados para su análisis.
    \item Apliqué análisis de redes en Python y técnicas econométricas para mapear el \textit{Product Space} y orientar la diversificación exportadora.
    \item Estimé la calidad  de productos mediante indicadores basados en valores unitarios, implementando los cálculos econométricos en Python.
    \item Desarrollé modelos gravitacionales en Python para proyectar exportaciones y evaluar escenarios de nearshoring con sustento estadístico.
    \item Cuantifiqué el impacto macroeconómico de distintos choques de exportación utilizando la Matriz Insumo--Producto combinada con herramientas econométricas en Python.
\end{itemize}
{\bf Asistente de investigación – Universidad de los Andes} \hfill {\em Mayo, 2024 – Actualidad}
\begin{itemize}
    \item Realicé análisis de datos y desarrollé modelos económetricos y de aprendizaje automático para estudiar temas relacionados con el crimen, en colaboración con el Prof. Daniel Mejía (PhD, Brown University).
    \item Ejecuté una consultoría para el Consejo Gremial Nacional como parte del Observatorio de Seguridad, en la cual:
    \begin{itemize}
        \item Actualicé y complementé una base de datos con los principales indicadores de seguridad nacional y regional, integrando fuentes oficiales y gremiales mediante Python y SQL.
        \item Elaboré informes mensuales para gremios y afiliados, con análisis de delitos e indicadores clave por región, y diseñé visualizaciones interactivas en Python.
        \item Contribuí a la redacción de documentos de política pública y recomendaciones técnicas a partir de estadísticas y hallazgos del observatorio.
    \end{itemize}
    \item Apliqué técnicas avanzadas de análisis estadístico y teoría de redes en proyectos de investigación bajo la supervisión del Prof. Tomás Rodríguez (PhD, Stanford University).
    \item Colaboré en la preparación de artículos académicos y presentaciones, incluyendo la visualización de datos complejos para ambos proyectos de investigación.
\end{itemize}

{\bf Data Analyst - Rappi} \hfill {\em Agosto, 2022 - Enero, 2024} 
%\vspace{2mm}
\begin{itemize}
    \item Generación de estadísticas descriptivas, gráficas y análisis estadísticos para comprender órdenes defectuosas en la vertical de operaciones, utilizando SQL (Snowflake), Python, Excel y Power BI. 
    \item Desarrollo de un modelo de procesamiento del lenguaje y análisis de texto utilizando Python. En este, se realizó la tokenización y vectorización del texto para realizar un modelo de clasificación Random Forest. %Se procesó +2.000 observaciones. Aumentó la visibilidad de los problemas usuarios evaluados en un 60\%.
    \item Desarrollo de un modelo para identificar casos de robo de productos, fraude de usuario y repartidores mediante el algoritmo de Levenshtein y tokenización de texto utilizando SQL y Python. \%Se procesaron +40 millones de observaciones. Ahorro estimado de +\$20.000 USD mensual en la operación.
\end{itemize}\\ 
{\bf Desarrollador - IBM} \hfill {\em Marzo, 2022 - Agosto, 2022} 
%\vspace{2mm}
\begin{itemize}
    \item Desarrollo de RPAs (Automatización Robótica de Procesos), utilizando IBM RPA Studio, con el objetivo de automatizar procesos de extracción y procesamiento de datos  en aplicaciones y páginas web.
    \item Experiencia internacional en equipos multiculturales y bilingües.
\end{itemize}\\ 
{\bf Data Science and Automation Intern - IBM} \hfill {\em Agosto, 2021 - Febrero, 2022} 
%\vspace{2mm}
\begin{itemize}
    \item Desarrollo de un proceso de ETL utilizando web scraping, utilizando Python, que procesó la data, 10 variables diferentes, de 700 mil productos de tres cadenas de supermercados diferentes. Este fue la base para que el equipo Senior realizara un análisis de mercado en el área de retail.
    \item Apoyo en la evaluación de una alianza con el Centro para la Cuarta Revolución Industrial a través de sintetización de información y documentación. 
\end{itemize}\\ 


\end{rSection}

\begin{rSection}{Educación}

{\bf Maestría en Economía} \hfill {\em Enero, 2024 - Actualmente} 
\\ Universidad de los Andes - Bogotá, Colombia \hfill 
{ GPA: 4.20/5.00}
\\ Facultad de Economía
\\
\\{\bf Economista} \hfill {\em 2018 - 2023} 
\\ Universidad de los Andes - Bogotá, Colombia \hfill { GPA: 4.08/5.00}
\\ Facultad de Economía
\end{rSection}

\begin{rSection}{PUBLICACIONES}

\textit{Fortalecimiento del SAT mediante Machine Learning en la detección de violencia atípica}
\\ \small{Tesis en procesos de maestría en Economía, Universidad de los Andes.}  
\vspace{-2mm}
\begin{tcolorbox}[colback=gray!5, colframe=black!20, boxrule=0.4pt, arc=2pt]
La investigación busca complementar y fortalecer el trabajo del Sistema de Alertas Tempranas (SAT) mediante herramientas de clasificación supervisada aplicadas a datos de violencia, conflicto armado y condiciones socioeconómicas a nivel municipal.
\end{tcolorbox}
\end{rSection}



\begin{rSection}{EXPERIENCIA PEDAGÓGICA}

{\bf Profesor de cátedra - Análisis de datos y Big Data} \hfill {\em Mayo, 2024 - Actualmente} 
\\ Universidad de los Andes - Educación continua.

\\
\\
{\bf Tutor de pregrado - Macroeconomía II} \hfill {\em Agosto, 2020 - Junio, 2022} 
\\ \textbf{Profesor:} David Peréz Reyna (PhD, University of Minnesota) y Andrés Zambrano (PhD, UCLA)
\\
\\
{\bf Tutor de pregrado - Economía laboral y hechos macroecónomicos} \hfill {\em Agosto, 2021 - Diciembre, 2021} 
\\ \textbf{Profesor:} Luis Eduardo Arango Thomas (PhD, University of Liverpool)

\end{rSection}

\begin{rSection}{ACTIVIDADES} \itemsep -2pt
{\bf Intercambio cultural y de idiomas} \hfill {\em 2020-2021} 
\\ \textit{Education First International Campus - San Diego, Estados Unidos}
\\
\\
{\bf Semillero de Macroeconomía} \hfill {\em 2020-2021} 
\\ \textit{Universidad de los Andes - Bogotá, Colombia}
\\
\\
{\bf Revista Supuestos} \hfill {\em 2020} 
\\ \textit{Universidad de los Andes - Bogotá, Colombia}

\end{rSection}

%----------------------------------------------------------------------------------------
\begin{rSection}{Habilidades}

{\bf Software:}\\  \textit{Avanzado:} Herramientas de ofimática (Google y Microsoft), Python, SQL
\\ \textit{Intermedio:} Matlab, R,  \LaTeX
\\ \textit{Básico:} Stata, Power BI (DAX)

\\ {\bf Idiomas:} \\ Español (Nativo), Inglés (C1)

%\\ {\bf Habilidades blandas:} \\ 
%Pensamiento critico, resolución de problemas, comunicación efectiva, %gestión del tiempo, organización.

\end{rSection}



\end{document}
