\documentclass{resume}

\usepackage[left=0.5in,top=0.5in,right=0.5in,bottom=0.6in]{geometry} % Document margins
\newcommand{\tab}[1]{\hspace{.2667\textwidth}\rlap{#1}}
\newcommand{\itab}[1]{\hspace{0em}\rlap{#1}}
\usepackage[most]{tcolorbox}

\makeatletter
\def \printname {
  \begingroup
    \hfil{\namesize\bfseries \@name}\hfil
    \nameskip\break
  \endgroup
}
\makeatother

% Contact information

\address{ \href{http://www.linkedin.com/in/juan-diego-heredia-nino}{\underline{{LinkedIn}}} $|$ +57 (316) 043\-5259 $|$  \href{http://jdheredian.github.io}{\underline{Web Page}} $|$ \href{mailto:juandiegoheredianino@gmail.com}{\underline{juandiegoheredianino@gmail.com}}}
\name{Juan Diego Heredia Niño} % Full name


\begin{document}
\begin{rSection}{Profile}
Master's student in Economics and Economist from Universidad de los Andes. Interests focused on the application of quantitative methods, such as statistics and machine learning, in both academic and operational research. Areas of interest include data analysis/science, applied macro/microeconomics, and machine learning. Skilled at working in teams in an organized way, solving problems under pressure, and handling large datasets with precision and attention to detail. Service-oriented with effective communication skills.
\end{rSection}

\begin{rSection}{Work Experience}
{\bf Research Assistant – Fedesarrollo} \hfill {\em March, 2025 – Present}
\begin{itemize}
    \item Conduct empirical research on the opportunities and challenges of nearshoring for Colombia in strategic export sectors.
    \item Built international trade databases (exports and imports since 1996) using UN COMTRADE and ITC (International Trade Centre) with Python.
    \item Developed automated web scraping tools to extract export projections from ITC's online tool, improving the efficiency of data collection and analysis.
    \item Created interactive visualizations and technical documents for public policy presentations.
\end{itemize}

{\bf Research Assistant – Universidad de los Andes} \hfill {\em May, 2024 – Present}
\begin{itemize}
    \item Performed data analysis and econometric modeling on crime-related topics under the supervision of Prof. Daniel Mejía (PhD, Brown University).
    \item Led a consulting project for the National Business Council as part of the Security Observatory, where:
    \begin{itemize}
        \item Updated and enriched a national and regional security indicators database, integrating official and sectoral sources using Python and SQL.
        \item Produced monthly reports for business associations, analyzing crime data and key indicators by region, and developed interactive Python visualizations.
        \item Contributed to the drafting of public policy documents and technical recommendations based on the observatory's findings.
    \end{itemize}
    \item Applied advanced statistical analysis and network theory in research projects under the supervision of Prof. Tomás Rodríguez (PhD, Stanford University).
    \item Supported the preparation of academic articles and presentations, including the visualization of complex data structures.
\end{itemize}

{\bf Data Analyst - Rappi} \hfill {\em August, 2022 - January, 2024} 
\begin{itemize}
    \item Generated descriptive statistics, visualizations, and statistical analysis to understand failed orders in the operations vertical using SQL (Snowflake), Python, Excel, and Power BI.
    \item Developed a natural language processing model using Python to classify text through tokenization and vectorization, applying a Random Forest algorithm.
    \item Designed a model to detect product theft, user fraud, and delivery scams using Levenshtein distance and text tokenization with SQL and Python.
\end{itemize}

{\bf Automation Developer - IBM} \hfill {\em March, 2022 - August, 2022} 
\begin{itemize}
    \item Developed Robotic Process Automation (RPA) solutions using IBM RPA Studio to automate data extraction and processing from web applications and sites.
    \item Gained international experience working in multicultural and bilingual teams.
\end{itemize}

{\bf Data Science and Automation Intern - IBM} \hfill {\em August, 2021 - February, 2022} 
\begin{itemize}
    \item Created an ETL process through web scraping using Python to process 10 variables across 700,000 products from three different supermarket chains. The cleaned dataset was used by the Senior team for market analysis in the retail sector.
    \item Supported the evaluation of a partnership with the Center for the Fourth Industrial Revolution through information synthesis and documentation.
\end{itemize}
\end{rSection}

\begin{rSection}{Education}

{\bf Master's in Economics} \hfill {\em January, 2024 – Present} 
\\ Universidad de los Andes - Bogotá, Colombia \hfill 
{ GPA: 4.20/5.00}
\\ Faculty of Economics
\\
\\{\bf Economist} \hfill {\em 2018 – 2023} 
\\ Universidad de los Andes - Bogotá, Colombia \hfill { GPA: 4.08/5.00}
\\ Faculty of Economics
\end{rSection}

%\begin{rSection}{Research Work}

%\textit{Strengthening the Early Warning System (SAT) through Machine Learning to Detect Atypical Violence}
%\\ \small{Master's thesis in progress – Universidad de los Andes}  
%\vspace{-2mm}
%\begin{tcolorbox}[colback=gray!5, colframe=black!20, boxrule=0.4pt, arc=2pt]
%The project aims to enhance the work of the Early Warning System (SAT) using supervised classification methods applied to violence, armed conflict, and socioeconomic data at the municipal level.
%\end{tcolorbox}
%\end{rSection}

\begin{rSection}{Teaching Experience}

{\bf Adjunct Lecturer – Data Analysis and Big Data} \hfill {\em May, 2024 – Present} 
\\ Universidad de los Andes – Continuing Education


{\bf Undergraduate Teaching Assistant – Macroeconomics II} \hfill {\em August, 2020 – June, 2022} 
\\ \textbf{Professors:} David Peréz Reyna (PhD, University of Minnesota) and Andrés Zambrano (PhD, UCLA)

{\bf Undergraduate Teaching Assistant – Labor Economics and Macroeconomic Facts} \hfill {\em August, 2021 – December, 2021} 
\\ \textbf{Professor:} Luis Eduardo Arango Thomas (PhD, University of Liverpool)

\end{rSection}

\begin{rSection}{Activities} \itemsep -2pt
{\bf Cultural and Language Exchange} \hfill {\em 2020–2021} 
\\ \textit{Education First International Campus – San Diego, United States}
\\
\\
{\bf Macroeconomics Research Group} \hfill {\em 2020–2021} 
\\ \textit{Universidad de los Andes – Bogotá, Colombia}
\\
\\
{\bf Supuestos Student Journal} \hfill {\em 2020} 
\\ \textit{Universidad de los Andes – Bogotá, Colombia}
\end{rSection}

\begin{rSection}{Skills}

{\bf Software:}\\  \textit{Advanced:} Office tools (Google \& Microsoft), Python, SQL
\\ \textit{Intermediate:} Matlab, R, \LaTeX
\\ \textit{Basic:} Stata, Power BI (DAX)

{\bf Languages:} \\ Spanish (Native), English (C1)

\end{rSection}
\end{document}
